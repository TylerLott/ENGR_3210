% !TEX TS-program = pdflatex
% !TEX encoding = UTF-8 Unicode

% This is a simple template for a LaTeX document using the "article" class.
% See "book", "report", "letter" for other types of document.

\documentclass[11pt]{article} % use larger type; default would be 10pt

\renewcommand{\thesubsection}{\thesection.\alph{subsection}}

\usepackage[utf8]{inputenc} % set input encoding (not needed with XeLaTeX)

%%% Examples of Article customizations
% These packages are optional, depending whether you want the features they provide.
% See the LaTeX Companion or other references for full information.

%%% PAGE DIMENSIONS
\usepackage{geometry} % to change the page dimensions
\usepackage{graphicx}
\geometry{a4paper} % or letterpaper (US) or a5paper or....
% \geometry{margin=2in} % for example, change the margins to 2 inches all round
% \geometry{landscape} % set up the page for landscape
%   read geometry.pdf for detailed page layout information

\usepackage{graphicx} % support the \includegraphics command and options

% \usepackage[parfill]{parskip} % Activate to begin paragraphs with an empty line rather than an indent

%%% PACKAGES
\usepackage[T1]{fontenc}
\usepackage{booktabs} % for much better looking tables
\usepackage{array} % for better arrays (eg matrices) in maths
\usepackage{paralist} % very flexible & customisable lists (eg. enumerate/itemize, etc.)
\usepackage{verbatim} % adds environment for commenting out blocks of text & for better verbatim
\usepackage{subfig} % make it possible to include more than one captioned figure/table in a single float
% These packages are all incorporated in the memoir class to one degree or another...

%%% HEADERS & FOOTERS
\usepackage{fancyhdr} % This should be set AFTER setting up the page geometry
\pagestyle{plain} % options: empty , plain , fancy
\renewcommand{\headrulewidth}{0pt} % customise the layout...
\lhead{}\chead{}\rhead{}
\lfoot{}\cfoot{\thepage}\rfoot{}

%%% SECTION TITLE APPEARANCE
\usepackage{sectsty}
\allsectionsfont{} % (See the fntguide.pdf for font help)
% (This matches ConTeXt defaults)

%%% ToC (table of contents) APPEARANCE
\usepackage[nottoc,notlof,notlot]{tocbibind} % Put the bibliography in the ToC
\usepackage[titles,subfigure]{tocloft} % Alter the style of the Table of Contents
\renewcommand{\cftsecfont}{\rmfamily\mdseries\upshape}
\renewcommand{\cftsecpagefont}{\rmfamily\mdseries\upshape} % No bold!

%%% END Article customizations

%%% The "real" document content comes below...

\title{MAE 3210  HW2}
\author{Tyler Lott}
%\date{} % Activate to display a given date or no date (if empty),
         % otherwise the current date is printed 

\begin{document}
\maketitle

\section*{Question 1}

\subsection*{a.}

\begin{center}
\includegraphics[width=.9\linewidth]{Q_1_plot.png} \\
The root appears to be around $x=1$
\end{center}


\subsection*{b.}

Bisection with $\epsilon_s = 10\%$, $ f(x) = 0$ at {\boldmath$ x = 0.9375 $}

\subsection*{c.}

False position with $\epsilon_s = 0.2\%$, $f(x) = 0$ at {\boldmath$x = .899687$}

\clearpage

\section*{Question 2}

\subsection*{a.}

\begin{center}
	\includegraphics[width=.9\linewidth]{Q_2_plot.png} \\
	The root appears to be around $x=1$
\end{center}

\subsection*{b.}

Bisection with $\epsilon_s = 2\%$,  $ f(x) = 0$ at {\boldmath$x = -0.4140625$}

\subsection*{c.}

False position with $\epsilon_s = 2\%$ , $f(x) = 0$ at {\boldmath$x = -0.4151665$}

\clearpage

\section*{Question 3}

Adjust the equation to be a function in terms of x (replaced m with x):
\begin{center}
{\boldmath{\Large$f(x) = {9.81x \over 15} * (1-e^{({-15 \over x}) 10}) - 36$ }}
\end{center}


\begin{flushleft}
False Position for $m$ with $\epsilon_s = 0.1\%$, $f(x) = 0$ at {\boldmath $x = 59.964457$}
\end{flushleft}

\clearpage

\section*{Question 4}

\subsection*{a.}

\begin{center}
	\includegraphics[width=.9\linewidth]{Q_4_plot.png} \\
	The root appears to be around $x=1$
\end{center}

\subsection*{b.}

Newton Raphson with $\epsilon_s = 0.01\%$ at $x_0 = 0$ found to be $ f(x) = 0$ at {\boldmath$x = 0.1338017$} \\
Newton Raphson with $\epsilon_s = 0.01\%$ at $x_0 = 3$ found to be $ f(x) = 0$ at {\boldmath$x = 3.210755$} \\
Newton Raphson with $\epsilon_s = 0.01\%$ at $x_0 = 4.5$ found to be $ f(x) = 0$ at {\boldmath$x = 4.6554423$}

\clearpage

\section*{Question 5}

\subsection*{a.}

\begin{center}
	\includegraphics[width=.9\linewidth]{Q_5_plot.png} \\
	The root appears to be around $x=1$
\end{center}

\begin{flushleft}
	Secant method $xl = -0.5$ and $xu = 0$, and $\epsilon_s = 0.1\%$ found to be $ f(x) = 0$ at {\boldmath$x = -0.416$} \\
	Secant method $xl = 1$ and $xu = 3$, and $\epsilon_s = 0.1\%$ found to be $ f(x) = 0$ at {\boldmath$x = 1.83$} \\
	Secant method $xl = 5$ , $xu = 6$, and $\epsilon_s = 0.1\%$  found to be $ f(x) = 0$ at{\boldmath$x = 5.25$} 
\end{flushleft}











\end{document}
